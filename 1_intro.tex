\chapter{Introduction}\label{chap:intro}
\epigraph{[Gig economy,] the collection of markets that match providers to consumers on a [job] basis in support of on-demand commerce. [\dots] Prospective clients request services through an Internet-based technological platform or smartphone application that allows them to search for providers or to specify jobs. Providers (gig workers) engaged by the on-demand company provide the requested service and are compensated for the jobs.}{Congressional Information Service\\\cite{Donovan2016a}}
\section{Information Design for Platform Drivers}
This thesis studies the relevance of providing demand information to platform drivers and how it should be allocated.

Platform drivers, as subset of all gig workers, include both grocery delivery drivers and ride-sourcing drivers. While food delivery, grocery delivery and ride-sourcing are distinct services, the groups share many similarities, and are sufficiently distinct from other sectors of the gig economy: Given demand information, both groups take on very short gigs that depend dynamically on demand by riders or food delivery customers. This dynamic nature makes information of different relevance to them compared to other parts of the gig economy such as accomodation (AirBnB, Vrbo) and freelancing (Amazon Mechanical Turk, Airtasker) which do not have to make such dynamic decisions because of the decreased time-sensitivity of their work. Also, many platforms operating in ridesourcing, so-called Transportation Network Companies (TNCs) such as Grab, Gojek, DiDi or Uber, offer also food and grocery delivery services.

The Congressional Information Service's definition of gig work given above highlights the opportunity of search for consumers: \enquote{Prospective clients request services through an Internet-based technological platform or smartphone application that allows them to \emph{search} for providers or to specify jobs} (emphasis added by the author). The definition does not include search by gig workers as a characteristic. As we argue for platform drivers, dynamically reacting, or \emph{searching and finding} demand is affecting driver earnings, and an important platform design parameter.

\subsection{Three Levels of Platform Design}
Information Design can be seen as a third level in the design of platforms.

\subsubsection{Matching}
Matching is the assignment of drivers to orders, i.e. rides, shopping trips, or deliveries. In the grander scheme of mechanism design, matching can be seen as the allocative dimension of ridehailing systems, and it answers the question of which driver is going to get work at all.
\subsubsection{Pricing}
Pricing refers to fares for customers and driver earnings for trips. It is the transfer domain in mechanism design.
\subsubsection{Information Design}
This thesis studies Information provision. While matching and pricing are real outcomes, behavior is shaped by forward-looking behavior and \emph{expectations}. Expectations are based on information. Informing drivers about when and where to work can affect their repositioning, their working hours, and location of their log-in, and hence help the platform.

Because of strategic responses by drivers, however, more information is not always better, neither for platforms nor drivers,. For example, if too many drivers get the information of a high expected demand in some area, this area might become congested, drivers might not get orders as frequently or might, because there is no supply-demand mismatch, not benefit from pricing strategies by platforms such as \emph{surge pricing}, compare \autoref{chap:literature}. This expectation leads drivers to less frequently visit high-demand areas.

The sophistication by drivers needed for such reasoning about demand has been demonstrated in other parts of the gig economy. For example, knowledge of demand and expectations have been demonstrated in the market for freelance software development \cite{Horton2019}, when, with the announcement of the abandonment of Flash by iOS, freelance software developers in this language migrated to other freelance work so that wages for Flash developers remained unchanged. 


After positioning platform drivers in the spheres of urban transportation and the gig economy in the rest of this chapter, and reviewing important dimensions of platform design which interact with information design and information provision through third parties in \autoref{chap:literature}, we make four contributions.

The first part of our study, \autoref{chap:jakarta} investigates which other factors besides expected earnings affect driver labor supply decisions. In a large survey of platform drivers in Jakarta, Indonesia, we find significant inconsistencies with earnings maximization of drivers in terms of reported earnings, and find cultural, as well as informational aspects in drivers' labor supply decisions. We use Natural Language Processing techniques to analyze free-text answers in a foreign language without translating answers pre-analysis, which might be of independent interest.

We supplement our observations on the relevance of information design questions in \autoref{chap:chengdu} by comparing the earnings effect of different driver repositioning strategies. We provide an approximate value of demand knowledge using earnings differences under optimal \emph{repositioning}, i.e. movement of idle drivers to other areas in the city. We make this estimate by inferring real earnings for platform driver trips undertaken for TNC DiDi in November 2016 in Chengdu, China and comparing the earnings differences to average earnings of delivery drivers on the platform. We find that drivers with very good demand information can earn significantly more than other drivers, as long as only few drivers have such information.

We relax our assumption that only a small group is repositioned in our theoretical analysis in \autoref{chap:info}. In a stylized information design model we show that only a few drivers receiving demand information is a robust property of optimal platform information design. We show that platforms that cannot commit to information policies to drivers have incentives to give demand information to many drivers, who, in turn act less on this information. We study solutions to this commitment problem via loyalty ranks, third-party aggregators, and audits.

The final \autoref{chap:policy} collects policy recommendations both for platforms such as Grab, GoJek, Didi, Uber or Lyft and regulators such as the New York City Taxi and Limousine Commission. We first argue that information design is of different significance for platforms and regulators. While the former will need to rely on third-party aggregators to commit to disclosing information in a way that drivers will not ignore, the latter should think more carefully about equity in the proposed models---Information needs to be provided asymmetrically on a per-trip basis, but symmetrically on an aggregate level. We conclude with engineering challenges arising from our analysis.

While we provide several facets of information design for platform drivers, we largely abstract both matching and pricing platform design. Real-world platforms solve joint optimization problems of pricing and information, and, as a further complication, often do so in competition with other platforms. While this means that our focus on information design underestimates the ability to design a market for platforms, we highlight that information design should be an additional pillar besides matching and pricing platform design.

Our analysis is also not restricted to human drivers. Depending on Autonomous Vehicle governance structure, conflicts of interest and congestion might have even more pronounced effect in a future of automated mobility. Our investigations, in particular in \autoref{chap:chengdu} and \autoref{chap:infodesign} directly feed also into this design question: Information Design has potentially large effects, and needs to be allocated thoughtfully to avoid congestion.
\section{Background}
Platform drivers take an increasingly important role in urban mobility (\autoref{subsec:pdandurbanmobility}), and have special characteristics compared to other forms of gig work (\autoref{subsec:pdandgigwork}). Some of these characteristics have led to more prominent regulatory interventions into platform driving (\autoref{subsec:pdregulation}) compared to other parts of the gig economy. Regulators have for the biggest part not intervened into information provision to drivers, and drivers find themselves in an informational environment between their platforms' apps and third parties aggregating information (\autoref{subsec:pdinfo}).
\subsection{Platform Drivers and Urban Mobility}\label{subsec:pdandurbanmobility}
As of 2021, services provided by platform drivers now make up $\nicefrac{1}{3}$ of the global taxi market \cite{Bryan2019a}. With the rise of the first TNCs about 15 years ago, the rapid changes leaves many of the design dimensions of this market open.

An important difference to other travel modes such as public transit is the need for dynamic rebalancing, i.e. ensuring a distribution of drivers such that in each part of the urban area expected ride demand can be met at each point in time, and rider waiting times are approximately constant throughout an urban area and time. While dynamic rebalancing is an important question in public transit operations, schedules can be designed with much less uncertainty as compared to the dynamic optimization required in TNC operations.

Part of the uncertainty making dynamic rebalancing for platforms hard comes from the need to design incentives that ensure participation from drivers. Even with accurate demand models for riders, platform drivers usually have the opportunity to log off their platform or reject (some of) the rides requested by a customer. If they are not incentivized to continue working on the platform, they might not.

\subsection{Platform Drivers as Gig Workers}\label{subsec:pdandgigwork}
Platform drivers are also gig workers \emph{qua} the definition introducing this chapter. Customers search on the platform and are matched to a driver for their ride or delivery. Nevertheless, platform drivers are extreme in some of their characteristics compared to other gig workers.

First, platform driving is, compared to other parts of the gig economy, very time-sensitive. This characteristic comes from the nature of deliveries or rides being short gigs. Compared to other parts of the gig economy, e.g. accomodation (e.g., on AirBnB), and services for care (e.g., care.com), technology (e.g., Andela), design (e.g., 99designs) and home services (e.g., Porch), whose gigs typically require scheduling or completion within at least a day, platform drivers have short gigs which are arranged within seconds. The shortness of gigs and scheduling makes dynamic driver incentive management particularly crucial for platform drivers.

Second, the terms of the contract are, more than in other parts of the gig economy, more strongly determined by the platform. While drivers have flexibility on when and where to work, which orders they will get, and also the pay they will receive, is not determined by them. This is in contrast to other platforms such as in accommodation, technology and design, where gig workers can set their own prices.

Third, reaching out to customers outside of the platform is much more challenging for platform drivers than for other gig workers. As argued above, compared to other gig work, platform driver gigs are short, and as both the complete contract and all payments are processed by the platform, there are few opportunities for drivers to get connection with riders that might lead to follow-up work. The absence of work options outside of platforms gives the platform additional bargaining power in negotiations with drivers.

Finally, many of the drivers are full-time and work many hours a day. This both raises concerns over work safety, but also over wage bounds, some of which have been introduced as regulations.

\subsection{Regulation of Platform Driver Work}\label{subsec:pdregulation}
The combination of time-sensitive gigs, pay determination by platforms, missing outside options for drivers, and full-time platform work led to policy interventions regarding minimum wages and the employment status of platform drivers.

The New York City Transport and Limousine Commission introduced an earnings standard which guaranteed a \emph{proxy} minimum wage for ridesourcing drivers in New York City \cite{Parrott2018}. Drivers earn for rides an additional amount to increase the average as calibrated against historical demand data to the minimum wage level.

In 2020, California accepted via a referendum a special provision for ride-sourcing drivers to be excluded from legal sufficient conditions for an employment relationship \cite{Padilla2020}. 

In contrast, the UK Supreme Court decided that ride-sourcing drivers \emph{are} employees of TNCs, with labor implications for drivers, with effects for payment and insurance of drivers \cite{Arden2021}. 

These regulations do not include provisions on the information that drivers get from TNCs. Drivers' information environment is given by different players, and interacts with drivers' payment structure.

\subsection{Platform Drivers' Payment and Information Environment}\label{subsec:pdinfo}
Most platform drivers interact with the platform through a smartphone app, which offers them potential gigs, which they can accept or reject. At each point in time, drivers can select to log off the system. Among U.S. driver, roughly half multi-home (\cite[p.48]{Valderrama2020}). In our survey conducted in Jakarta (\autoref{chap:jakarta}), we find that a smaller number of drivers multi-home in this market, and that even fewer have several apps open at the same time. Hence, many drivers rely in getting demand information on one platform's app, as well as driver assistants. 

Driver assistants such as the Surge app or Gridwise give information on demand across platforms and allow to track earnings information. These are outside of the contractual information, and purely provide drivers with information.

The informational environment becomes relevant for drivers through the payment structure of gig drivers. An important abstraction will be three time periods, which we call, in accordance with nomenclature introduced by Uber \emph{phases 1, 2, and 3}, compare \autoref{fig:uberperiods}. 

\begin{figure}
\includegraphics[width=\linewidth]{Figures/periods.png}
\caption{Nomenclature for different periods in coverage for Uber as presented in \cite{Inc.2021}.}	\label{fig:uberperiods}
\end{figure}

Period 0 corresponds to times off the platform. Period 1 corresponds to times where the driver is online but idle. In these times, drivers are not paid.

Phase 2 is after the acceptance of an order, and \emph{en route} to fulfilling the order. Except for flat payments for cancellations, drivers are not paid for this part of their trips as well. 

Period 3 refers to the time between the acceptance of a ride request and the pickup of the passenger, and the ride, respectively. In these phases, riders are paid according to a combination of base prices, minutes and distance driven, as well as surge multipliers.

The goal of this thesis is to establish the relevance of providing drivers with demand information, and will only significantly affect driver behavior in phase 1. Drivers that are picking up or driving a rider have a clear task, and even a contract for their work. The main question we ask in our theoretical section will be under which the platform can shape behavior outside times where a contract between the platform driver and the customer is in place.