\chapter{Platform Driver Labor Supply Beyond Earnings Maximization}\label{chap:jakarta}
\begin{quote}
Since Uber’s introduction in 2009, ridesharing platforms like Uber, Didi, Grab, and Lyft have radically transformed the taxi and limo industry. These services, which allow consumers to order a car to their location via a smartphone application, now control roughly 1/3 of the international taxi market. In other words, a ridesharing firm acts as a platform matching drivers to riders and setting the pricing terms between them. ---\cite{Bryan2019a}
\end{quote}
 This chapter analyzes data from a survey in Jakarta and shows that drivers do not purely maximize their earnings in their labor supply choices. In an analysis of free-text answers, we highlight understanding of the app and information as important determinants of driver behavior.
\section{Survey}
Our data comes from a survey conducted in spring 2021 in Jakarta, Indonesia.
\subsection{Sample}
Participants were asked to complete an online survey hosted on Qualtrics. 846 complete survey responses were submitted. We excluded from the analysis drivers that stated more than 16 hours of work per day and stated earnings smaller than 10,000 IRB or larger than 2,000,000 IRB. The survey participants were recruited through social media (platform driver WhatsApp groups).

Figure \ref{fig:age} shows the age distribution of participants, where around 50\% of drivers are within the age group of 30 to 39. Figure \ref{fig:gender_mode_driver} reveals that over 97\% of drivers who participated in the survey are male and around 97\% are motorbike (delivery) drivers. 

Three major ride-hailing companies operate in Jakarta: Grab, GoJek and Shopee. While Grab and GoJek are incumbents in this market, Shoppee entered the market only in 2021. Figure \ref{fig:platform} shows the percentage of drivers working for each of the platforms, double-counting multi-homers. Around 94\% of drivers work for the incumbents Grab or GoJek. In our sample of drivers, only 12\% of drivers questioned in the survey are multi-homers.
\begin{figure}[!ht]
    \centering
    \includegraphics[width=10cm]{platform.png}
    \caption{Platform distribution of drivers. (GoJek: 60\%, Grab: 34\%, Shopee: 6\%)}
    \label{fig:platform}
\end{figure}  
 \begin{figure}[!ht]
      \centering
      \begin{subfigure}[b]{0.49\textwidth}
          \centering
          \includegraphics[width=\textwidth]{age.png}
          \caption{Age distribution of drivers. (Below 20: 2\%, 20 to 29: 27\%, 30 to 39: 49\%, 40 to 49: 20\%, above 50: 3\%)}
          \label{fig:age}
      \end{subfigure}
      \hfill
      \begin{subfigure}[b]{0.49\textwidth}
          \centering
          \includegraphics[width=\textwidth]{gmd.png}
          \caption{Gender, mode and driver type distributions of drivers. (Male: 97\%, female: 3\%; motorbike: 97\%, car: 3\%; single-homer: 88\%, multi-homer: 12\%) }
          \label{fig:gender_mode_driver}
      \end{subfigure}
         \caption{Sociodemographic information and driving behaviors of survey participants.}
         \label{fig:survey_stats}
 \end{figure}
\subsection{Questions}
 The survey consists of multiple questions blocks and is reproduced in its entirety in Appendix~\ref{appendix:survey}. 

In a first block, participants were asked to check all TNCs they have worked for in the past 30 days. A survey respondent is defined as a \emph{multi-homer} if they have been working for at least two platforms in the past 30 days. In this first block, basic information and driving behavior are asked as well, e.g., vehicle type used, previous occupation, behaviors while waiting for the next order.

The second block's questions differ for multi-homers and non-multi-homers. Multi-homers are asked how they switch between different platforms, allocations of their working hours and reasons for multi-homing. For non-multi-homers, reasons for non-multi-homing and factors that would make them multi-home are elicited.

In a third block, for each ride-hailing platform survey participants have worked for in the last 30 days, respondents are asked questions about their labor supply choices. Type of work (delivery, ridesourcing), part of the city of highest activity, and daily working hours, distance, and earnings.

A final block elicited socio-demographic information, including age, gender, educational status, income level, living areas, and weekly expenses. 
\section{Descriptive Evidence}
A first observation we make is that there is significant correlation between experience working as a taxi driver before and multi-homing behaviors. The OLS regression model shown in Table \ref{tab:opang} suggests that drivers who has been motorcycle taxi drivers before are less likely to multi-home in Jakarta market.
 \begin{table}
 \centering
     \caption{Results of regression models predicting multi-homing behaviors of ride-hailing drivers in Jakarta based on previous occupation}
         \begin{tabular}{lcccccc}
         \toprule
         \textbf{Variable} & \textbf{coef} & \textbf{std err} & \textbf{t} & \textbf{P$> |$t$|$} & \textbf{[0.025} & \textbf{0.975]}  \\
         \midrule
         \textbf{Intercept} &       0.1276***  &        0.012     &    10.735  &         0.000        &        0.104    &        0.151     \\
         \textbf{Former taxi driver}     &      -0.0796**  &        0.031     &    -2.574  &         0.010        &       -0.140    &       -0.019     \\
         \midrule
         \multicolumn{1}{r}{\textbf{R-squared: }}   & 0.008 & \multicolumn{3}{c}{\textbf{Adjusted R-squared: }}  & 0.007 & \\
         \bottomrule 
         \end{tabular}
     \label{tab:opang}
 \end{table}

 Finally, we turn our attention to the results of our regression models exploring the impact of being a part-time ride-hailing driver on drivers' multihoming behaviors.
 
 From the model results shown in Table \ref{tab:part-timer}, we find that part-time ride-hailing drivers are more likely to multi-home in our sample. However, it is a less significant predictor of multihoming behaviors compared to other variables above.

 \begin{table}
 \centering
     \caption{Results of regression models predicting multi-homing behaviors of ride-hailing drivers in Jakarta based on whether being part-time ride-hailing drivers}
         \begin{tabular}{lcccccc}
         \toprule
         \textbf{Variable} & \textbf{coef} & \textbf{std err} & \textbf{t} & \textbf{P$> |$t$|$} & \textbf{[0.025} & \textbf{0.975]}  \\
         \midrule
         \textbf{Intercept}                           &       0.1097***  &        0.012     &     9.258  &         0.000        &        0.086    &        0.133     \\
         \textbf{Part-Time} &       0.0441  &        0.032     &     1.384  &         0.167        &       -0.018    &        0.107     \\
         \midrule
         \multicolumn{1}{r}{\textbf{R-squared: }}   & 0.002 & \multicolumn{3}{c}{\textbf{Adjusted R-squared: }}  & 0.001 & \\
         \bottomrule 
         \end{tabular}
     \label{tab:part-timer}
 \end{table}
 \subsection{Earnings for Different Platforms}
Our main finding concerns the earnings for different platforms. Figure \ref{fig:hourly_salary} displays the hourly salary distribution of all survey respondents in Jakarta. The median respondent makes less than 20,000 IRB (approximately 1.4 USD) per hour, which is significantly less than the average hourly salary 120,175 IRB (approximately 8.4 USD) in Jakarta~\cite{Jakarta_hourly_salary}. This stands even when controlling for sociodemographics.

 \begin{figure}
     \centering
     \includegraphics[width=0.6\textwidth]{hourlysalary.png}
     \caption{Hourly salary distribution of survey respondents (ride-hailing drivers in Jakarta)}
     \label{fig:hourly_salary}
 \end{figure}

We estimate, among all single-homers, the correlation between working for GoJek and hourly salary in Table \ref{tab:hourly_salary}. We exclude the 3\% of non-motorcycle drivers due to their significantly different earnings. The model results suggest that GoJek drivers earn significantly less than drivers from other platforms, 11,000 IRB (approximately 0.77 USD) less per hour. While a weak effect ($R^2 = 0.002$), it is significant.

\begin{table}[!htbp] \centering
\begin{tabular}{@{\extracolsep{5pt}}lcc}
\\[-1.8ex]\hline
\hline \\[-1.8ex]
& \multicolumn{2}{c}{\textit{Dependent variable:}} \
\cr \cline{2-3}
\\[-1.8ex] & (1) & (2) \\
\hline \\[-1.8ex]
 Intercept & 30713.817$^{***}$ & 32141.158$^{***}$ \\
  & (1927.295) & (10919.276) \\
 Q54[T.30 - 39] & & 599.344$^{}$ \\
  & & (9829.024) \\
 Q54[T.30 - 39]:Q58[T.Yes] & & -1459.150$^{}$ \\
  & & (10215.505) \\
 Q54[T.40 - 49] & & -5300.705$^{}$ \\
  & & (13705.572) \\
 Q54[T.40 - 49]:Q58[T.Yes] & & 2649.267$^{}$ \\
  & & (14135.068) \\
 Q54[T.Above 50] & & -655.081$^{}$ \\
  & & (17545.153) \\
 Q54[T.Above 50]:Q58[T.Yes] & & 3628.347$^{}$ \\
  & & (19003.614) \\
 Q54[T.Under 20] & & -689.077$^{}$ \\
  & & (4819.356) \\
 Q54[T.Under 20]:Q58[T.Yes] & & -689.077$^{}$ \\
  & & (4819.356) \\
 Q55[T.Male] & & -5813.134$^{}$ \\
  & & (4617.561) \\
 Q55[T.Prefer not to answer] & & -17966.204$^{}$ \\
  & & (19349.857) \\
 Q57[T.None of the above] & & -0.000$^{}$ \\
  & & (0.000) \\
 Q57[T.S1 or higher] & & -5956.143$^{}$ \\
  & & (7460.034) \\
 Q57[T.SD] & & 6745.220$^{}$ \\
  & & (8008.309) \\
 Q57[T.SMA] & & -338.772$^{}$ \\
  & & (5669.506) \\
 Q57[T.SMP] & & 1058.908$^{}$ \\
  & & (6598.707) \\
 Q58[T.Yes] & & 5170.436$^{}$ \\
  & & (8691.783) \\
 gojek[T.True] & -11001.520$^{***}$ & -10506.619$^{***}$ \\
  & (2366.886) & (2426.682) \\
\hline \\[-1.8ex]
 Observations & 552 & 552 \\
 $R^2$ & 0.038 & 0.052 \\
 Adjusted $R^2$ & 0.036 & 0.025 \\
 Residual Std. Error & 26284.803(df = 550) & 26435.066(df = 536)  \\
 F Statistic & 21.605$^{***}$ (df = 1.0; 550.0) & 1.942$^{**}$ (df = 15.0; 536.0) \\
\hline
\hline \\[-1.8ex]
\textit{Note:} & \multicolumn{2}{r}{$^{*}$p$<$0.1; $^{**}$p$<$0.05; $^{***}$p$<$0.01} \\
\end{tabular}
\caption{Results of regression models predicting hourly earnings of platform drivers drivers in Jakarta based on whether working for GoJek}\label{tab:working_hours}
\end{table}


For the same group of drivers, we consider the difference of total working hours between ride-hailing platforms.
 The model results in Table \ref{tab:working_hours} show that GoJek drivers work significantly longer than drivers from other platforms, 1.0335 hours more per day.
 Both regression results imply that there existing a large switching cost between GoJek and other platforms (Grab and Shopee) in Jakarta ride-hailing market. 
 
 GoJek motorcycle drivers work longer hours but earn less compared to other drivers. We use Natural Language Processing tools to gain additional insights on what, besides earnings maximization, might drive the choices to work for Grab as compared to GoJek.

 
 
 \begin{table}[!htbp] \centering
\begin{tabular}{@{\extracolsep{5pt}}lcc}
\\[-1.8ex]\hline
\hline \\[-1.8ex]
& \multicolumn{2}{c}{\textit{Dependent variable:}} \
\cr \cline{2-3}
\\[-1.8ex] & (1) & (2) \\
\hline \\[-1.8ex]
 Intercept & 10.664$^{***}$ & 8.328$^{***}$ \\
  & (0.162) & (0.874) \\
 Q54[T.30 - 39] & & -0.761$^{}$ \\
  & & (0.791) \\
 Q54[T.30 - 39]:Q58[T.Yes] & & 0.970$^{}$ \\
  & & (0.829) \\
 Q54[T.40 - 49] & & -1.880$^{*}$ \\
  & & (1.072) \\
 Q54[T.40 - 49]:Q58[T.Yes] & & 1.916$^{*}$ \\
  & & (1.116) \\
 Q54[T.Above 50] & & -1.056$^{}$ \\
  & & (1.440) \\
 Q54[T.Above 50]:Q58[T.Yes] & & 1.595$^{}$ \\
  & & (1.572) \\
 Q54[T.Under 20] & & 1.204$^{}$ \\
  & & (2.642) \\
 Q54[T.Under 20]:Q58[T.Yes] & & -2.903$^{}$ \\
  & & (2.785) \\
 Q55[T.Male] & & 0.928$^{**}$ \\
  & & (0.395) \\
 Q55[T.Prefer not to answer] & & 1.378$^{}$ \\
  & & (1.523) \\
 Q57[T.None of the above] & & -0.000$^{}$ \\
  & & (0.000) \\
 Q57[T.S1 or higher] & & -0.198$^{}$ \\
  & & (0.603) \\
 Q57[T.SD] & & 0.073$^{}$ \\
  & & (0.698) \\
 Q57[T.SMA] & & -0.032$^{}$ \\
  & & (0.480) \\
 Q57[T.SMP] & & 0.063$^{}$ \\
  & & (0.568) \\
 Q58[T.Yes] & & 1.749$^{**}$ \\
  & & (0.689) \\
 gojek[T.True] & 1.033$^{***}$ & 0.866$^{***}$ \\
  & (0.210) & (0.206) \\
\hline \\[-1.8ex]
 Observations & 655 & 655 \\
 $R^2$ & 0.036 & 0.136 \\
 Adjusted $R^2$ & 0.034 & 0.114 \\
 Residual Std. Error & 2.648(df = 653) & 2.536(df = 638)  \\
 F Statistic & 24.120$^{***}$ (df = 1.0; 653.0) & 6.273$^{***}$ (df = 16.0; 638.0) \\
\hline
\hline \\[-1.8ex]
\textit{Note:} & \multicolumn{2}{r}{$^{*}$p$<$0.1; $^{**}$p$<$0.05; $^{***}$p$<$0.01} \\
\end{tabular}
\caption{Results of regression models predicting daily working hours of ride-hailing drivers in Jakarta based on whether working for GoJek}\label{tab:hourly_salary}
\end{table}

\section{Free-Text}
 In this section, we use a topic model (Latent Dirichlet Analysis) to discover potential non-monetary reasons for the choice of TNC that drivers work for. All of the analysis is use answers to question 16 (compare Appendix~\ref{appendix:survey}) \enquote{Anything else we should know about why you decided not to use multiple platforms?}.

We fine-tuned a natural language model BERT \cite{Devlin2019} trained on the Indonesian Wikipedia to predict the choice of platform worked for based on the answer to question 16. We list global feature importance in our training data set for the prediction of the platform in \autoref{tab:free-text}.

The words most indicative of working for GoJek, the platform with lower reported earnigns, are

\begin{table}
\begin{tabular}{ccc}
\toprule
&&\\
\midrule
&&\\
\bottomrule	
\end{tabular}
\caption{Sum of Feature Importance in SHAP scores}	
\end{table}


%TODO: FInish analysis.
%Potential things to see:
%gridwise: Using across platforms
%vehicle requirements
%bounced out
%focus/confusion
%indonesian incumbent
\section{Discussion}
We find significant earnings and working hours differences between single-homing drivers. The majority of respondents works for a platform where they report longer working hours and less hourly wages.

In our analysis of free-text answers, we find that drivers working for a lower-paying platform cite more frequently reasons for %TODO: Finish this.

This gives introductory evidence for features beyond matching and pricing, relating to the informational environment of drivers. The value of demand information, whose design we are going to study in \autoref{chap:infodesign} is part of our analysis the next chapter.
 \section{Conclusion}
