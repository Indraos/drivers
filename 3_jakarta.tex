\chapter{Non-Monetary Shapers of Supply Decisions}\label{chap:jakarta}
\begin{quote}
    Since Uber’s introduction in 2009, ridesharing platforms like Uber, Didi, Grab, and Lyft have
radically transformed the taxi and limo industry. These services, which allow consumers to order
a car to their location via a smartphone application, now control roughly 1/3 of the international
taxi market. In other words, a ridesharing firm acts as a platform matching drivers to riders and
setting the pricing terms between them. ---\cite{Bryan2019a}
\end{quote}
%TODO: Incorporate this structure
%- Study
%- Time, place, seeding
%- sociodemographics
%- questions are in appendix
%
%- Findings
%- Eardnings Difference
%- Institutional Persistence
%- Free-text answers
%
%- Conclusions
%- There are factors beyond earnings-maximization
%- And pricing and allocation seem not to be referred to
%- Broader perspective on design of platforms could help




 This chapter analyzes results from a survey in Jakarta and shows significant departures from earnings maximization among platform drivers in Indonesia. We highlight context in free-text answers to gather evidence for reasons outside of earnings maximization.
\section{Survey}
We analyze data from a survey conducted in spring 2021 in Jakarta, Indonesia.
\subsection{Sample}
Participants were asked to complete an online survey hosted on Qualtrics and received 846 complete survey responses after being distributed and circulated to drivers through known social media (WhatsApp groups) used by drivers.

Figure \ref{fig:age} shows the age distribution of participants, where around 50\% of drivers are within the age group of 30 to 39. Figure \ref{fig:gender_mode_driver} reveals that over 97\% of drivers who participated in the survey are male and around 97\% are motorbike drivers. 

There exist three major ride-hailing companies in Jakarta: Grab, GoJek and Shopee. While Grab and Gojek are incumbents in this market, Shoppee entered the market only in 2021. Figure \ref{fig:platform} shows the percentage of drivers working for each of the platforms, double-counting multi-homers. Around 94\% of drivers work for the incumbents Grab or GoJek. In our sample of drivers, only 12\% of drivers questioned in the survey are multi-homers.

We are not aware of demographics of TNC drivers in Jakarta and hence unable to verify representativity of our sample.
\begin{figure}[!ht]
    \centering
    \includegraphics[width=10cm]{platform.png}
    \caption{Platform distribution of drivers. (GoJek: 60\%, Grab: 34\%, Shopee: 6\%)}
    \label{fig:platform}
\end{figure}  
 \begin{figure}[!ht]
      \centering
      \begin{subfigure}[b]{0.49\textwidth}
          \centering
          \includegraphics[width=\textwidth]{age.png}
          \caption{Age distribution of drivers. (Below 20: 2\%, 20 to 29: 27\%, 30 to 39: 49\%, 40 to 49: 20\%, above 50: 3\%)}
          \label{fig:age}
      \end{subfigure}
      \hfill
      \begin{subfigure}[b]{0.49\textwidth}
          \centering
          \includegraphics[width=\textwidth]{gmd.png}
          \caption{Gender, mode and driver type distributions of drivers. (Male: 97\%, female: 3\%; motorbike: 97\%, car: 3\%; single-homer: 88\%, multi-homer: 12\%) }
          \label{fig:gender_mode_driver}
      \end{subfigure}
         \caption{Sociodemographic information and driving behaviors of survey participants.}
         \label{fig:survey_stats}
 \end{figure}
%
\subsection{Questions}
 The survey consists of multiple questions blocks and is given in its entirety in Appendix~\ref{appendix:survey}. 

In a first block, participants were asked to check all TNCs they have been working for in the past 30 days. A survey respondent is defined as a \emph{multi-homer} if they have been working for at least two platforms in the past 30 days. In this first block, basic information and driving behavior are asked as well, e.g., vehicle type used, previous occupation, behaviors while waiting for a next ride/order.

The second block depends on whether respondents were identified in the first block as multi-homers or not. Multi-homers are asked how they switch between different platforms, allocations of their working hours and reasons for multi-homing. For non-multi-homers, reasons for non-multi-homing and factors that would make them multi-home are asked.

In a third block, for each ride-hailing platform survey participants have worked for in the last 30 days, we asked a number of questions about driving activities during the last two weeks.
These questions included whether drivers mostly worked on delivering food or transporting passengers, their most frequent driving area, days and hours to work, total driving distance, and daily salaries.

A final block elicited socio-demographic characteristics including age, gender, educational status, income level, living areas, and weekly expenses. 

\section{Findings}
 To understand the multi-homing behavior and who works for multiple ride-hailing platforms, we first constructed a Lasso regression model with the regularization coefficient $0.001$ based on drivers' socio-demographic information, which were shown in Table~\ref{tab:multihoming}.%TODO CHoosing \alpha independent of the regression.
 We find that male drivers whose age is between 30 to 39 and above 50 are less likely to multi-home in Jakarta market. Also, drivers are less likely to multi-home if working for ride-hailing platforms is the main source of their income.
 On the other hand, we find that drivers with higher degrees are significantly more likely to multi-home. 
 \begin{table}
 \centering
     \caption{Results of regression models predicting multi-homing behaviors of ride-hailing drivers in Jakarta based on sociodemographic information}
         \begin{tabular}{lcccccc}
         \toprule
         \textbf{Variable} & \textbf{coef} & \textbf{std err} & \textbf{t} & \textbf{P$> |$t$|$} & \textbf{[0.025} & \textbf{0.975]}  \\
         \midrule
         \textbf{Intercept}                   &       0.3168***  &        0.052     &     6.040  &         0.000        &        0.214    &        0.420     \\
         \textbf{Age (30 - 39)}              &      -0.0455*  &        0.025     &    -1.786  &         0.075        &       -0.095    &        0.005     \\
         \textbf{Age (above 50)}             &      -0.1189*  &        0.069     &    -1.721  &         0.086        &       -0.255    &        0.017     \\
         \textbf{Male}                 &      -0.1213***  &        0.042     &    -2.911  &         0.004        &       -0.203    &       -0.040     \\
         \textbf{Degree (bachelor's or higher)}         &       0.1258***  &        0.044     &     2.840  &         0.005        &        0.039    &        0.213     \\
         \textbf{Degree (junior high school)}                  &       0.0327  &        0.035     &     0.939  &         0.348        &       -0.036    &        0.101     \\
         \textbf{\makecell[l]{Ride-hailing as main \\ source of income} }          &      -0.0733*  &        0.038     &    -1.925  &         0.055        &       -0.148    &        0.001     \\
         \textbf{\makecell[l]{Age (40 - 49) and ride-hailing \\as main source of income}} &      -0.0480  &        0.033     &    -1.474  &         0.141        &       -0.112    &        0.016     \\
         \midrule
         \multicolumn{1}{r}{\textbf{R-squared: }}   & 0.033 & \multicolumn{3}{c}{\textbf{Adjusted R-squared: }}  & 0.025 & \\
         \bottomrule 
         \end{tabular}
     \label{tab:multihoming}
 \end{table}
 Next, we observe the correlation between previous work as a motorcycle taxi driver before on multi-homing behaviors. The OLS regression model shown in Table \ref{tab:opang} suggests that drivers who has been motorcycle taxi drivers before are less likely to multi-home in Jakarta market.
 \begin{table}
 \centering
     \caption{\textbf{Results of regression models predicting multi-homing behaviors of ride-hailing drivers in Jakarta based on previous occupation}}
         \begin{tabular}{lcccccc}
         \toprule
         \textbf{Variable} & \textbf{coef} & \textbf{std err} & \textbf{t} & \textbf{P$> |$t$|$} & \textbf{[0.025} & \textbf{0.975]}  \\
         \midrule
         \textbf{Intercept} &       0.1276***  &        0.012     &    10.735  &         0.000        &        0.104    &        0.151     \\
         \textbf{Motorcycle taxi driver before}     &      -0.0796**  &        0.031     &    -2.574  &         0.010        &       -0.140    &       -0.019     \\
         \midrule
         \multicolumn{1}{r}{\textbf{R-squared: }}   & 0.008 & \multicolumn{3}{c}{\textbf{Adjusted R-squared: }}  & 0.007 & \\
         \bottomrule 
         \end{tabular}
     \label{tab:opang}
 \end{table}

 Finally, we turn our attention to the results of our regression models exploring the impact of being a part-time ride-hailing driver on drivers' multihoming behaviors.
 From the model results shown in Table \ref{tab:part-timer}, we find that part-time ride-hailing drivers are more likely to multihome in our sample. However, it is a less significant predictor of multihoming behaviors compared to other variables above.

 \begin{table}
 \centering
     \caption{\textbf{Results of regression models predicting multi-homing behaviors of ride-hailing drivers in Jakarta based on whether being part-time ride-hailing drivers}}
         \begin{tabular}{lcccccc}
         \toprule
         \textbf{Variable} & \textbf{coef} & \textbf{std err} & \textbf{t} & \textbf{P$> |$t$|$} & \textbf{[0.025} & \textbf{0.975]}  \\
         \midrule
         \textbf{Intercept}                           &       0.1097***  &        0.012     &     9.258  &         0.000        &        0.086    &        0.133     \\
         \textbf{Being part-time ride-hailing drivers} &       0.0441  &        0.032     &     1.384  &         0.167        &       -0.018    &        0.107     \\
         \midrule
         \multicolumn{1}{r}{\textbf{R-squared: }}   & 0.002 & \multicolumn{3}{c}{\textbf{Adjusted R-squared: }}  & 0.001 & \\
         \bottomrule 
         \end{tabular}
     \label{tab:part-timer}
 \end{table}
 \subsection{Earnings for Different Platforms}
 Then, we consider the Jakarta ride-hailing market and drivers' hourly salary and total working hours. Figure \ref{fig:hourly_salary} displays the hourly salary distribution of all survey respondents in Jakarta, which fits into a negative binomial distribution. Most ride-hailing drivers earn less than 20,000 Rp (approximately 1.4 USD) per hour, which is significantly less than the average hourly salary 120,175 Rp (approximately 8.4 USD) in Jakarta~\cite{Jakarta_hourly_salary}.

 \begin{figure}
     \centering
     \includegraphics[width=0.6\textwidth]{hourlysalary.png}
     \caption{Hourly salary distribution of survey respondents (ride-hailing drivers in Jakarta)}
     \label{fig:hourly_salary}
 \end{figure}

 To understand whether drivers working for different ride-hailing platforms are payed differently, we constructed a regression model on the impact of working for GoJek on hourly salary, shown in Table \ref{tab:hourly_salary}. In this model, we only consider motorcycle drivers who work exclusively for one platform. 
 The model results suggest that GoJek drivers earn significantly less than drivers from other platforms, 11,000 Rp (approximately 0.77 USD) less per hour.

 \begin{table}
 \centering
     \caption{\textbf{Results of regression models predicting hourly salary of ride-hailing drivers in Jakarta based on whether working for GoJek}}
         \begin{tabular}{lcccccc}
         \toprule
         \textbf{Variable} & \textbf{coef} & \textbf{std err} & \textbf{z} & \textbf{P$> |$z$|$} & \textbf{[0.025} & \textbf{0.975]}  \\
         \midrule
         \textbf{Intercept}     &    30710  &     2215.544     &    13.863  &         0.000        &     26400    &     35100    \\
         \textbf{Only GoJek} &     -11000  &     2547.137     &    -4.319  &         0.000        &     -16000    &    -6009.224     \\
         \midrule
         \multicolumn{1}{r}{\textbf{R-squared: }}   & 0.038 & \multicolumn{3}{c}{\textbf{Adjusted R-squared: }}  & 0.036 & \\
         \bottomrule 
         \end{tabular}
     \label{tab:hourly_salary}
 \end{table}

 Furthermore, we built another regression model to investigate the difference of total working hours between ride-hailing platforms.
 The model results in Table \ref{tab:working_hours} show that GoJek drivers work significantly longer than drivers from other platforms, 1.0335 hours more per day.
 Both regression results imply that there existing a large switching cost between GoJek and other platforms (Grab and Shopee) in Jakarta ride-hailing market. 
 GoJek motorcycle drivers work longer hours but earn less compared to other drivers.

 \begin{table}
 \centering
     \caption{\textbf{Results of regression models predicting daily working hours of ride-hailing drivers in Jakarta based on whether working for GoJek}}
         \begin{tabular}{lcccccc}
         \toprule
         \textbf{Variable} & \textbf{coef} & \textbf{std err} & \textbf{t} & \textbf{P$> |$t$|$} & \textbf{[0.025} & \textbf{0.975]}  \\
         \midrule
         \textbf{Intercept}     &      10.6642***  &        0.162     &    65.928  &         0.000        &       10.347    &       10.982     \\
         \textbf{Only GoJek} &       1.0335***  &        0.210     &     4.911  &         0.000        &        0.620    &        1.447     \\
         \midrule
         \multicolumn{1}{r}{\textbf{R-squared: }}   & 0.036 & \multicolumn{3}{c}{\textbf{Adjusted R-squared: }}  & 0.034 & \\
         \bottomrule 
         \end{tabular}
     \label{tab:working_hours}
 \end{table}

\subsection{Free-Text}
 In this section, we use a topic model (Latent Dirichlet Analysis) to discover potential non-monetary reasons for the choice of TNC that drivers work for. All of the analysis is use answers to question 16 (compare Appendix~\ref{appendix:survey}) \enquote{Anything else we should know about why you decided not to use multiple platforms?}.

 In a topic analysis, the word \enquote{focus} is highly descriptive of one of the clusters. Respondents indicate that \enquote{focus was needed for best results} and that they would not like to \enquote{confuse the apps}. 

 Others refer to the character of one of the platforms as being an Indonesian incumbent Gojek that competes with an (in our sample) better-paying incumbent from Singapore. Respondents make reference to \enquote{children of the nation} to underline their choices.

 We use survey evidence showing that in an  a significant share of non-multi homing drivers in an oligopolistic market for ridehailing in Jakarta. Our findings hint at substantial frictions to multi-homing, non-monetary incentives in the decision what TNC drivers choose to drive for, and when to drive.

\subsection{Discussion}
 As our evidence can only explain a small part of the variation in regression models, we propose other variables that could explain more of the variation in multi-homing and pay.
 \paragraph{Experience}
 Respondents mention \emph{focus} prominently in reasons for multi-homing. To disambiguate whether this focus refers to learning of the platform or on permanent presence on the platform, joining date of the platform could be used as a proxy for experience.
 \paragraph{Risk Preferences}
 Another important explanation of variation in earnings could be given by the risk preferences of drivers. High variation in pay could stem from different behavior and different willingness to drive in times where getting a job from the platform are uncertain. Existing short surveys \cite{falk2016preference} allow to estimate risk preferences.
 
 \section{Conclusion}
