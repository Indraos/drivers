\chapter{Related Work}\label{chap:literature}
Our work at the intersection of the Labor Economics in the Gig Economy, Supply Management in Transportation, Information Design, and information gathering from platforms contributes to several literatures.
\section{Estimating Labour Supply Through Surveys}
Our contribution first contribution, a survey to platform drivers in Jakarta, Indonesia, uses a survey to study the labor supply decisions of drivers, and does so using a survey.
\subsection{Models of Platform Driver Labour Supply}
Labor supply choices of drivers, or more generally participation decisions in two-sided platforms, is assumed as exogenous, or rational expectations of drivers is assumed.

Papers showing the impact of multi-homing use either stylized models such as those inspired by Hotelling lines \cite{bryan2019theory,bakos2020platform}, assume exogenous models of \cite{castillo2017surge} or do not model multi-homing decisions \cite{liu2017impact}.

Other papers assume that in moments where agents join a platform, they have exact knowledge of the surplus that is gained and the prices that both sides receive \cite{liu2019multihoming}.

\cite{Wang_Yang_2019} offers a comprehensive review on the ride-hailing system including demand, supply and platform perspectives. 
Driver behavior is modeled differently in the ride-hailing literature.   
Regarding operations and optimizations for ride-hailing platforms, e.g., driver-customer matching, idle vehicle rebalancing and vehicle routing, drivers are assumed to follow the instructions of platforms~(\cite{Bertsimas_Jaillet_Martin_2019, Alonso-Mora_2017, Braverman_Dai_Liu_Ying_2019, Wen_Zhao_Jaillet_2017}).

Simple driver behavior models are proposed in the pricing literature of the ride-hailing system. 
\cite{Bai_So_Tang_Chen_Wang_2019} propose a queuing model to optimize platforms' profit while considering price-sensitive customers and earning-sensitive drivers. Each driver has a reservation earning rate based on his outside option and the driver will provide services only if earning rate exceeds the reserved value.
\cite{Taylor_2018} evaluate the impact of uncertainty in passenger delay sensitivity and driver independence on platforms' optimal per-service price and wage. Each driver has an opportunity cost and the driver will participate in the platform when having non-negative expected utility. 

To better understand driver behavior, several studies discuss the general reasons why drivers work in the ride-hailing system.
An econometric framework with closed-form measures is proposed by \cite{Sun_Wang_Wan_2019} to estimate both the participation elasticity and working-hour elasticity for drivers. In particular, this article assumes that there 
\cite{Chaudhari_Byers_Terzi_2018} design an earnings-maximization strategy for drivers and show that strategic behavior about when and where to drive have significant impact on drivers' earnings.
\cite{Jiang_Kong_Zhang_2021} show that regret aversion and ignorance of suggestion are the two major behavioral factors which influence drivers' re-positioning decisions.
\subsection{Labour Economics and Industrial Economics}
Our analysis in \autoref{chap:jakarta} pertains to 

\section{Investigations into Platform Pricing}
To understand platform pricing, many platforms 
\subsection{Scraping-Based Studies}
Some platforms scrape data from platforms, such as studies on platform competition in New York \tocite (rosaia) and a consumer surplus comparison of taxi and ridesourcing \tocite (shapiro). Similarly, \tocite (peaking under the hood) uses large-scale simulation of apps.
\subsection{Information from Platforms}
\subsection{Gathering Driver Data}
\section{Driver Repositioning}
The repositioning of drivers, also known as Dynamic Rebalancing in the literature on platform design. 
\section{Cheap Talk and Information Design}
Our third contribution contributes to a line of work on cheap talk and information design. 
\subsection{Cheap Talk}
The standard model of Crawford-Sobel \tocite, in which one agent, which has a piece of information, communicates with another agent, who can take an action affecting both agents' utility, too strong mismatch of incentives leads to no communication. A classical model of cheap talk, communication is costless, and does neither binds sender nor receiver to take a particular action in the future \tocite (Farrell). In our first model of information provision, we model the strategic interaction between the platform and drivers as a cheap talk game, and show that if the incentives of the platform and drivers diverge sufficiently, that, in equilibrium, no information is transmitted, in terms of the cheap talk literature, the platform \emph{babbles}.
\subsection{Bayesian Persuasion and Information Design}
In contrast to cheap talk, Information Design models allow the sender to commit to rules for which information they reveal \tocite (kamenica gentzkow, also annual reviews). This can have profound implications for the outcome compared to Cheap Talk. This will also be a feature in our model in \autoref{chap:infodesign}. 

\subsection{Information and Mechanism Co-Design}
While we consider a pure mechanism design problem, work on the co-design of mechanisms and information, such as \tocite(mechanism design with aftermarkets). This paper considers the informational effect of allocations. In our platform driver setting this would model the informational effect of surge prices for rational drivers.