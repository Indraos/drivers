\begin{abstractpage}
With the rise of app-based matching platforms, gig workers become important suppliers of labor to transportation and food delivery. Where and when these workers---platform drivers---work depends on their expected earnings, and hence, in turn, on their information on demand. This thesis gathers evidence why information on demand is important for driver labor supply and in which ways it should be allocated. We make four contributions.

First, using a large-scale survey of platform drivers in Jakarta, Indonesia, we document significant deviations from earnings maximization and extract, using Natural Language Processing techniques, driver's reasons for their labor supply choices. By adopting a fine-tuned transformer model, we are able to analyze Indonesian free-text answers, which might be applicable in other scenarios as well.

Second, we estimate the potential earnings effect of optimal information on demand in Chengdu, China. We use a repositioning challenge that asked for algorithms to reposition drivers with the goal to maximize their average earnings rate. The dataset given gives the earnings rate in artificial units, not in terms of RMB. We recover, using public information on driver earnings structure and regression analysis, the earnings rate equivalents in terms of RMB and, as an application, recover the inter-temporal distribution of surge prices.

Third, we introduce a theoretical model to highlight qualitative features of information design to platforms taking into account the roles of commitment of the platform to particular information policies and asymmetric information, i.e. information that only some agents receive. We show that both platform and driver payoffs are the highest under asymmetric information that the platform can commit to, that both are decreased under no commitment and might, if driving to a high-demand location is costly enough result in no information being transmitted. The inferior result is if the platform is restricted to only public information, in which case commitment is irrelevant. We show that in cases where only a subset of drivers is targeted by public information, efficiency can (approximately) be restored. We comment on two challenges, a fairness and a commitment challenge, arising for the platform.

We close this thesis by assembling challenges and opportunities for platforms, regulators and transportation engineers.
\end{abstractpage}