\chapter{Information Design}\label{chap:infodesign}
\begin{quote}
	The information design problem has
a literal interpretation: there
really is an information designer (or mediator, or sender) who can commit to provide extra information to players to serve her own
interests. While the commitment assumption may be problematic in many settings, it provides a useful benchmark.--\cite{Bergemann}
\end{quote}

%TODO: Incorporate this structure
%- Model 
%- Atomic Game
%- Cheap Talk Stage
%- BCE stage
%- Private vs. public information
%
%- Analysis
%
%- Private Information
%- Solution for Private Information Cheap Talk
%Private Information is desired
%- Solution for Private Information Info Design
%Babbling
%- Comparison
%
%- Public Information
%- Solution for Public Info Cheap Talk
%- Solution for Public Info Information Design
%- Comparison
%
%- Remedies
%
%- Commitment
%- Third Party Information Intermediaries
%- Repeated Game
%- Other sources of Commitment
%
%- Small group of targeted agents
%- Result
%- Third Party Interpretation
%- High-Scoring Agents Information
%
%- Conclusions


In this section, we study \emph{pure} information design for platform drivers. We refer to it as \emph{pure} as we study information provision while keeping the driver payoffs fixed. We show several limitations for the platform: Drivers, in equilibrium, ignore information if it cannot commit to not giving a particular driver some part of the information, it gives
\section{Model}
In our stylized model, two agents simultaneously decide whether to drive to a part of a city (\enquote{go}) or to follow other business on or off the platform (\enquote{not go}). The platform observes a demand state $\theta \in \{0,1\}$. We can view this as the outcome of an accurate demand prediction model. Before each of the drivers makes her decision on whether to drive, the system can send a message $m_1(\theta)$ to driver $1$ and $m_2(\theta)$ to driver $2$ on whether there is demand on the platform. Drivers get utility $1$ if they get an order, and incur a cost of driving to the location of $\varepsilon$. This gives, depending on the demand state, the game tables in \autoref{fig:twodrivers}

The platform gets utility of $1$ if at least one of the drivers gets to the desired location, and $0$ utility otherwise. The divergence of platform and driver utility stems from the payment structure for drivers, as we described in the introduction: Drivers only get paid for phase 2, when they are matched to a ride.

The platform can send messages to drivers. The platform can either send a message to drivers without committing to any particular form of information revelation (\autoref{subsec:infodesign}) such as \enquote{I recommend you to go to this place, and I will not recommend this to others}, or does not have the power to credibly commit to this (\autoref{subsec:cheaptalk}). 

\begin{figure}
\centering
\begin{subfigure}{.48\linewidth}
\begin{tikzpicture}
\matrix[matrix of math nodes,every odd row/.style={align=right},every even row/.style={align=left},every node/.style={text width=1.5cm},row sep=0.2cm,column sep=0.2cm] (m) {
$\frac12-\varepsilon$&$\sigma$\\
$\frac12-\varepsilon$&$1-\varepsilon$\\
$1-\varepsilon$&$\sigma$\\
$\sigma$&$\sigma$\\
};
\draw (m.north east) rectangle (m.south west);
\draw (m.north) -- (m.south);
\draw (m.east) -- (m.west);
\coordinate (a) at ($(m.north west)!0.25!(m.north east)$);
\coordinate (b) at ($(m.north west)!0.75!(m.north east)$);
\node[above=5pt of a,anchor=base] {go};
\node[above=5pt of b,anchor=base] {not go};

\coordinate (c) at ($(m.north west)!0.25!(m.south west)$);
\coordinate (d) at ($(m.north west)!0.75!(m.south west)$);
\node[left=2pt of c,text width=1cm]  {go};
\node[left=2pt of d,text width=1cm]  {not go};

\node[above=18pt of m.north] (firm b) {Driver 2};
\node[left=1.6cm of m.west,rotate=90,align=center,anchor=center] {Driver 1};
\end{tikzpicture}
\caption{Demand, $\theta = 1$}
\end{subfigure}
\begin{subfigure}{.48\linewidth}
\begin{tikzpicture}
\matrix[matrix of math nodes,every odd row/.style={align=right},every even row/.style={align=left},every node/.style={text width=1.5cm},row sep=0.2cm,column sep=0.2cm] (m) {
$-\varepsilon$&$\sigma$\\
$-\varepsilon$&$-\varepsilon$\\
$-\varepsilon$&$\sigma$\\
$\sigma$&$\sigma$\\
};
\draw (m.north east) rectangle (m.south west);
\draw (m.north) -- (m.south);
\draw (m.east) -- (m.west);
\coordinate (a) at ($(m.north west)!0.25!(m.north east)$);
\coordinate (b) at ($(m.north west)!0.75!(m.north east)$);
\node[above=5pt of a,anchor=base] {go};
\node[above=5pt of b,anchor=base] {not go};

\coordinate (c) at ($(m.north west)!0.25!(m.south west)$);
\coordinate (d) at ($(m.north west)!0.75!(m.south west)$);
\node[left=2pt of c,text width=1cm]  {go};
\node[left=2pt of d,text width=1cm]  {not go};

\node[above=18pt of m.north] (firm b) {Driver 2};
\node[left=1.6cm of m.west,rotate=90,align=center,anchor=center] {Driver 1};
\end{tikzpicture}
\caption{No, $\theta = 0$}
\end{subfigure}
\caption{Payoff Matrix for the two-driver game.}\label{fig:twodrivers}
\end{figure}
We assume that the drivers best respond to the messages sent by the platform. In our first model, we assume that the platform can commit to a particular mapping from demand to messages, in the latter they cannot.


\subsection{Information Design}\label{subsec:infodesign}
We first study a setting where the platform can commit to an arbitrary mapping of demand $\theta$ to tuple of messages $(m_1(\theta), m_2 (\theta))$. Before stating our theorem, we go into public or private information.
\subsection{Public and Private Messages}
Some information is shown to all drivers, while other is shown only to a few drivers. We call messages \emph{public} if $m_1(\theta) = m_2(\theta)$ for $\theta = 0,1$ and all other messages \emph{private}. As is well-known in the information design literature \cite{Bergemann}, the game that agents play is one of \emph{strategic substitutes}, andp private information will increase the total utility for all players. In our case, this means, drivers. Our theorem adds that this is also the optimal choice of messages for the platform.
\begin{thm}\label{thm:commitment}
All maximizers of platform revenue are $m_1(\theta) = \theta$, 
\end{thm}


\subsection{Cheap Talk}\label{subsec:cheaptalk}
In cheap talk, the platform discloses some information to a driver, which, given the update, updates their behavior. More concretely, there is an abstract set of \emph{messages} $m \in \mathcal M$ that the platform can send. The players receive the message and play optimally. In particular, the timeline of the game is:
\begin{enumerate}
	\item The demand $\theta$ at the area is realized.
	\item The platform decides to send messages $(m_1, m_2)$ to the drivers.
	\item The drivers decide to drive or not drive to the area.
	\end{enumerate}
We solve for perfect Bayesian (signalling) equilibria of this game.
\begin{thm}\label{thm:commitment}
	In the cheap talk model, Perfect Bayesian equilibria have the following form:
	\begin{enumerate}
		\item If $\sigma < \frac12 - \varepsilon$ or $\sigma > 1-\varepsilon$, then the platform is indifferent between any message, the drivers will visit their dominant choice.
		\item If $\frac12 -\varepsilon \le \sigma  \le 1-\varepsilon$, the platform is indifferent between any message, and the drivers ignore the information. 
	\end{enumerate}
\end{thm}
We show a proof to this statement in \autoref{app:ommited}
If the general outside option is very low ($\sigma < \frac12 - \varepsilon$) or very high ($\sigma > 1 - \varepsilon$), drivers go to the area in hope to get a ride or stay away from it, respectively, even without demand information. The platform is indifferent between any demand information given that it won't influence the driver's decisions. This is an example where agents know that there is a high 

In the other cases, the platform cannot transmit any information. The 

In this environment, hence, if the not internalized cost from 

\section{Inefficiencies}
After analysing the predictions, we identify two sources of inefficiencies: a potential lack of commitment (a comparison between our cheap talk and the commitment solution) and a potential lack of being able to provide asymmetric information.
\subsection{Inefficiencies through lack of commitment}
\begin{prop}\label{cor:commitmentlack}
	The cheap talk version of the game has lower welfare. This statement is uniform: Both platform and driver surplus are lower in the cheap talk game.
\end{prop}
The statement shows that neither side benefits from limited commitment of the platform. The drivers have to rely on less information, and cannot trust the platform. The equilibrium is \enquote{babbling}. 

\subsection{Inefficiencies through public information}
Comparing our second two theorems, we find the following loss from public information.
\begin{cor}
In the commitment regime, only allowing public information reduces platform and driver surplus. In the cheap talk regime, this restriction increases welfare
\end{cor}
This result might look somewhat surprising.
\section{Discussion and Remedies}
Given the inefficiencies observed in the last section, some remedies for the market should be considered. We introduce two potential solutions to the platform's commitment problem, and one solution to the allocation problem. 
\subsection{Commitment via third parties}
A main driver of the inefficiencies in our model was that there is a conflict of incentives between the platform and the drivers. In this environment, if drivers \enquote{pay} the platform, there is not much. 
\subsection{Commitment via reputation}
A second opportunity for the platform to commit is via reputation. This assumes that the game of information provision studied in this section is repeated sufficiently often. 
\begin{thm}[\cite{Friedman1973}]
	In an infinitely repeated game with sufficiently patient agents, all individually rational payoffs can be achieved as a result of equilibrium play.
\end{thm}
This statement is allowed by agents playing a \enquote{punishment} equilibrium from the platform. It is unlikely in the setting of platform drivers that they can individually punish the platform, which, in return, does not incentivize the platform to give recommendations according to a commitment problem.
\subsection{Approximate Efficiency of Public Information to Few Drivers}
For our last statement, we will need a more general model. There are $n$ drivers, which learn a state $\theta \in \{0,1 , 2,\dots, n\}$. Drivers get a utility $1-\varepsilon$ if matched to a ride and they get a utility of $-\varepsilon$ if not matched. We assume that $n$ is large, and a small fraction of drivers can get a public message.
\begin{prop}
	Assume that drivers are 
\end{prop}