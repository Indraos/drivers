\chapter{Information Design}\label{chap:infodesign}
\epigraph{I think everyone just has the same mentality. Everyone just goes there, if there's surge there}{Anonymous platform driver in the Netherlands\\as quoted in \cite{Ashkrof2020}}

In this section, we study \emph{pure} information design for platform drivers. We refer to it as \emph{pure} as we study information provision while keeping the driver payoffs fixed. We show several limitations for the platform: Drivers, in equilibrium, ignore information if it cannot commit to not giving a particular driver some part of the information, it gives
\section{Model}
We present here, in a self-contained manner, a model of information design with an \emph{omniscient} designer. Omniscience refers to the TNC, the information designer, having strictly more information than all drivers, the agents in our game. This assumption is not without loss, as existing challenges to improve Operations Research models, e.g., in last-mile logistics, compare \cite{Last2007}, show. Our results, qualitatively do not differ. 

 We consider a finite set of agents $i = 1, 2, \dots, n$. There are states of the world $\Theta$. We consider a basic game given by, for each player $i$ an action set $A_i$ and a utility function
\[
u_i \colon A \times \Theta \to \R
\]
where $A = A_1 \times \cdots \times A_n$ and a prior distribution $F \in \Delta(\Theta)$ which we assume has full support. We assume that this prior is shared by all players and the designer. $((A_i, u_i)_{i=1,2,\dots, n}, F)$ hence specifies a standard game. 

Our main model will consider $n=2$ drivers that simultaneously decide whether to drive to a part of a city (\enquote{go}) or to follow other business on or off the platform (\enquote{not go}), i.e. $A_1 = A_2 = \{\text{go}, \text{not go}\}$. The utility functions are given by the tables in \autoref{fig:twodrivers}. We assume that the prior is that with probability $l \in [0,1]$ demand is realized. This corresponds to an outside option for the drivers of $\sigma$ and a cost to go to the high-demand area of $\varepsilon$.

We call this game the \emph{basic game}. 

The information designer has a utility function
\[
v \colon A \times \Theta \to \R.
\] 
In our example, the information designer is a TNC. It will be an information designer, and will get utility $1$ if at least one driver is able to satisfy demand, otherwise a utility of $0$. 

The platform can designs an \emph{information structure} for each driver. This can be an arbitrary piece of information that drivers take into account. Under the assumption that the platform is omniscient and can commit to a rule, a \emph{revelation}-type argument can be used to simplify the design space: For any messages that the designer sends to agents, given that the platform knows the best response of the agents to this information, the platform can recommend the agents their best responses. Hence, any information structure has a particularly simple structure as \emph{recommendations}
\[
\sigma \colon \Theta \mapsto A.
\]
For a formal statement and a proof of the revelation function, see  \cite{Baskerville2011}.

The timeline of the model is:
\begin{enumerate}
	\item The information designer commits to an information structure $S = \sigma$.
	\item The state of the world is realized.
	\item The players each receive their action recommendations $\sigma_i(\theta)$.
	\item The agents select their action $a_i$.
	\item The payoffs are realized.
\end{enumerate}
\begin{figure}
\centering
\begin{subfigure}{.48\linewidth}
\begin{tikzpicture}
\matrix[matrix of math nodes,every odd row/.style={align=right},every even row/.style={align=left},every node/.style={text width=1.5cm},row sep=0.2cm,column sep=0.2cm] (m) {
$\frac12-\varepsilon$&$\sigma$\\
$\frac12-\varepsilon$&$1-\varepsilon$\\
$1-\varepsilon$&$\sigma$\\
$\sigma$&$\sigma$\\
};
\draw (m.north east) rectangle (m.south west);
\draw (m.north) -- (m.south);
\draw (m.east) -- (m.west);
\coordinate (a) at ($(m.north west)!0.25!(m.north east)$);
\coordinate (b) at ($(m.north west)!0.75!(m.north east)$);
\node[above=5pt of a,anchor=base] {go};
\node[above=5pt of b,anchor=base] {not go};

\coordinate (c) at ($(m.north west)!0.25!(m.south west)$);
\coordinate (d) at ($(m.north west)!0.75!(m.south west)$);
\node[left=2pt of c,text width=1cm]  {go};
\node[left=2pt of d,text width=1cm]  {not go};

\node[above=18pt of m.north] (firm b) {Driver 2};
\node[left=1.6cm of m.west,rotate=90,align=center,anchor=center] {Driver 1};
\end{tikzpicture}
\caption{Demand, $\theta = 1$}
\end{subfigure}
\begin{subfigure}{.48\linewidth}
\begin{tikzpicture}
\matrix[matrix of math nodes,every odd row/.style={align=right},every even row/.style={align=left},every node/.style={text width=1.5cm},row sep=0.2cm,column sep=0.2cm] (m) {
$-\varepsilon$&$\sigma$\\
$-\varepsilon$&$-\varepsilon$\\
$-\varepsilon$&$\sigma$\\
$\sigma$&$\sigma$\\
};
\draw (m.north east) rectangle (m.south west);
\draw (m.north) -- (m.south);
\draw (m.east) -- (m.west);
\coordinate (a) at ($(m.north west)!0.25!(m.north east)$);
\coordinate (b) at ($(m.north west)!0.75!(m.north east)$);
\node[above=5pt of a,anchor=base] {go};
\node[above=5pt of b,anchor=base] {not go};

\coordinate (c) at ($(m.north west)!0.25!(m.south west)$);
\coordinate (d) at ($(m.north west)!0.75!(m.south west)$);
\node[left=2pt of c,text width=1cm]  {go};
\node[left=2pt of d,text width=1cm]  {not go};

\node[above=18pt of m.north] (firm b) {Driver 2};
\node[left=1.6cm of m.west,rotate=90,align=center,anchor=center] {Driver 1};
\end{tikzpicture}
\caption{No, $\theta = 0$}
\end{subfigure}
\caption{Payoff Matrix for the two-driver game.}\label{fig:twodrivers}
\end{figure}
 Mathematically, the remaining constraint of \emph{obedience} can be written as
\begin{multline*}
	\sum_{a_i, \theta} u_i((a_i, a_{-i}); \theta)\sigma((a_i,a_{-i}) | \theta) F(\theta) \\\ge \sum_{a_i, \theta} u_i((a_i', a_{-i}); \theta)\sigma((a_i,a_{-i}) | \theta) F(\theta)
\end{multline*}
for any $a_i' \in A_i$. A profile of decision rules $\sigma$ that is obedient is called a Bayes Correlated Equilibrium (BCE). $\sigma$ is a coarse correlated equilibrium if an agent that updates their information on the state of the world given their recommendation does not want to deviate from said recommendation.

The platform's maximizes a utility function
\begin{equation}
V(\sigma) = \sum_{a,t,\theta} v(a,\theta) \sigma(a| \theta) F(\theta).\label{eq:value}
\end{equation}
The platform's optimal information design is maximizing \eqref{eq:value} among all BCEs $\sigma$.

We make the assumption that getting a ride, even after driving to a high-demand area is better than the outside option, but the outside option is preferrable to to getting a ride with $\frac12$ probability, $1-\varepsilon \ge \sigma \ge \frac12-\varepsilon$.

\begin{prop}
An optimal information design for the platform is given by
\begin{table}
\begin{subtable}{.49\linewidth}
\centering
\begin{tabular}{ccc}
\toprule
$\theta = 0$ & go & not go \\
\midrule
go & $\sigma$ & x \\
not go & x & 0\\
\bottomrule
\end{tabular}
\end{subtable}
\begin{subtable}{.49\linewidth}
\centering
\begin{tabular}{ccc}
\toprule
$\theta = 0$ & go & not go \\
\midrule
go & $\sigma$ & x \\
not go & x & 0\\
\bottomrule
\end{tabular}
\end{subtable}
\caption{Optimal Information Design}	
\end{table}
\end{prop}
Conditional on the good state, this minimizes the correlation of 


\begin{proof}
	Observe that every recommendation $\sigma$ that is not symmetric, i.e. $\sigma_1(\theta) \not \stackrel{d}{=} \sigma_2(\theta)$ for some $\theta$, because the game is symmetric, also $\sigma'$, $\sigma_2' = \sigma_1$, $\sigma_1' = \sigma_2$, is a Bayes-correlated equilibrium, and so is $\nicefrac{(\sigma + \sigma')}{2}$. Therefore, an optimal information design for the platform can be chosen to be symmetric. 
	
	Hence, the optimal information structure is characterized by the probability the probability $r_\theta$ that both drivers drive to the area and by $p_\theta$, the probability that each of the drivers goes. This gives rise to the information structure in \autoref{tab:paraminfodesign}. 
	\begin{table}
\begin{subtable}{.49\linewidth}
\centering
\begin{tabular}{ccc}
\toprule
$\theta = 0$ & go & not go \\
\midrule
go & $r_0$ & $p_0 - r_0$\\
not go & $p_0 - r_0$ & $1 + r_0 - 2p_0$\\
\bottomrule
\end{tabular}
\end{subtable}
\begin{subtable}{.49\linewidth}
\centering
\begin{tabular}{ccc}
\toprule
$\theta = 0$ & go & not go \\
\midrule
go & $r_1$ & $p_1 - r_1$\\
not go & $p_1 - r_1$ & $1 + r_1 - 2p_1$\\
\bottomrule
\end{tabular}
\end{subtable}
\caption{Parameterized Information Design}\label{tab:paraminfodesign}	
\end{table}
For all entried in \autoref{tab:paraminfodesign} to be probabilities, it needs to hold that 
\[
	\max\{0,2p_\theta - 1\} \le r_\theta \le p_\theta
\]
For $a = \text{go}$, the obedience constraint is
\[
lp_0\sigma 
\]
The platform would like to minimize the probability that the demand cannot be met, i.e. minimize $l(1-p_B)^2 + (1-l)(1-p_B)^2$.
\end{proof}


\subsection{Information Deisgn}\label{subsec:infodesign}
We compare this to a model in which the platform cannot commit ot 
In cheap talk, the platform discloses some information to a driver, which, given the update, updates their behavior. More concretely, there is an abstract set of \emph{messages} $m \in \mathcal M$ that the platform can send. The players receive the message and play optimally. In particular, the timeline of the game is:
\begin{enumerate}
	\item The demand $\theta$ at the area is realized.
	\item The platform decides to send messages $(m_1, m_2)$ to the drivers.
	\item The drivers decide to drive or not drive to the area.
	\end{enumerate}
We solve for perfect Bayesian (signalling) equilibria of this game.

\subsection{Cheap Talk}
We call a recommendation $\sigma$ public if $\sigma_1 (\theta)\stackrel{\text{a.s.}}{=} \sigma_2 (\theta)$ for $\theta = 0,1$,i.e. if both agents receive the same information. In this case, this leads to the following result:
\begin{prop}
The optimal information design when the platform is restricted to public messages is given in \autoref{tab:optpublic}	
\end{prop}
	\begin{table}
\begin{subtable}{.49\linewidth}
\centering
\begin{tabular}{ccc}
\toprule
$\theta = 0$ & go & not go \\
\midrule
go & $r_0$ & $p_0 - r_0$\\
not go & $p_0 - r_0$ & $1 + r_0 - 2p_0$\\
\bottomrule
\end{tabular}
\end{subtable}
\begin{subtable}{.49\linewidth}
\centering
\begin{tabular}{ccc}
\toprule
$\theta = 0$ & go & not go \\
\midrule
go & $r_1$ & $p_1 - r_1$\\
not go & $p_1 - r_1$ & $1 + r_1 - 2p_1$\\
\bottomrule
\end{tabular}
\end{subtable}
\caption{Parameterized Information Design}\label{tab:paraminfodesign}	
\end{table}

\subsection{Private and Public Messages}\label{subsec:cheaptalk}
\begin{thm}\label{thm:commitment}
	In the cheap talk model, Perfect Bayesian equilibria have the following form:
	\begin{enumerate}
		\item If $\sigma < \frac12 - \varepsilon$ or $\sigma > 1-\varepsilon$, then the platform is indifferent between any message, the drivers will visit their dominant choice.
		\item If $\frac12 -\varepsilon \le \sigma  \le 1-\varepsilon$, the platform is indifferent between any message, and the drivers ignore the information. 
	\end{enumerate}
\end{thm}
We show a proof to this statement in \autoref{app:ommited}
If the general outside option is very low ($\sigma < \frac12 - \varepsilon$) or very high ($\sigma > 1 - \varepsilon$), drivers go to the area in hope to get a ride or stay away from it, respectively, even without demand information. The platform is indifferent between any demand information given that it won't influence the driver's decisions. This is an example where agents know that there is a high 

In the other cases, the platform cannot transmit any information. The 

In this environment, hence, if the not internalized cost from 

\section{Main Result}
\begin{thm}\label{cor:commitmentlack}
	The optimal information design is given by
	


	The optimal cheap talk is given by
	
	
	The driver surplus, as well as the platform surplus are, in equilibrium, increased
\end{thm}
\section{Remedies and Extensions}
Given the inefficiencies observed in the last section, some remedies for the market should be considered. We introduce two potential solutions to the platform's commitment problem, and one solution to the allocation problem. 
\subsection{Commitment via third parties}
A main driver of the inefficiencies in our model was that there is a conflict of incentives between the platform and the drivers. In this environment, if drivers \enquote{pay} the platform, there is not much. To show this part, we amend 
\subsection{Commitment via reputation}
A second opportunity for the platform to commit is via reputation. This assumes that the game of information provision studied in this section is repeated sufficiently often. 
\begin{thm}[\cite{Friedman1973}]
	In an infinitely repeated game with sufficiently patient agents, all individually rational payoffs can be achieved as a result of equilibrium play.
\end{thm}
This statement is allowed by agents playing a \enquote{punishment} equilibrium from the platform. It is unlikely in the setting of platform drivers that they can individually punish the platform, which, in return, does not incentivize the platform to give recommendations according to a commitment problem.
\subsection{Approximate Efficiency of Public Information to Few Drivers}
For our last statement, we will need a more general model. There are $n$ drivers, which learn a state $\theta \in \{0,1 , 2,\dots, n\}$. Drivers get a utility $1-\varepsilon$ if matched to a ride and they get a utility of $-\varepsilon$ if not matched. We assume that $n$ is large, and a small fraction of drivers can get a public message.
\begin{prop}
	Assume that drivers are 
\end{prop}