\chapter{Information Design}\label{chap:infodesign}
\begin{quote}
	The information design problem has
a literal interpretation: there
really is an information designer (or mediator, or sender) who can commit to provide extra information to players to serve her own
interests. While the commitment assumption may be problematic in many settings, it provides a useful benchmark.--\tocite
\end{quote}
In this section, we study pure information design for platform drivers. We refer to it as \emph{pure} as we study information provision while keeping the pay to drivers fixed. Part of our conclusions in this section are that for sufficiently high non-internalized costs from the platform, pure information design will lead to no information transmission to drivers in equilibrium.
\section{Model}
In our stylized model, two agents simultaneously decide to drive to a part of a city or to do something else. The platform observes, or has a better estimate on the demand state $\theta \in \{0,1\}$. Before each of the drivers makes her decision on whether to drive, the system can send a message $m_1$ to driver $1$ and $m_2$ to driver $2$ on whether there is demand on the platform. Drivers get utility $1$ if they get an order, and incur a cost of driving to the location of $\varepsilon$. This gives a game table in \autoref{fig:twodrivers}. 

The platform gets utility of $1$ if at least one of the drivers gets to the desired location, and $0$ utility otherwise. The divergence of platform and driver utility stems from the payment structure for drivers, as we described in the introduction: Drivers only get paid for phase 2, when they are matched to a ride.

The platform can send messages to drivers. The platform can either send a message to drivers without committing to any particular form of information revelation (\autoref{subsec:infodesign}) such as \enquote{I recommend you to go to this place, and I will not recommend this to others}, or does not have the power to credibly commit to this (\autoref{subsec:cheaptalk}). 

\begin{figure}
\centering
\begin{tikzpicture}
\matrix[matrix of math nodes,every odd row/.style={align=right},every even row/.style={align=left},every node/.style={text width=1.5cm},row sep=0.2cm,column sep=0.2cm] (m) {
$\frac12-\varepsilon$&$\sigma$\\
$\frac12-\varepsilon$&$1-\varepsilon$\\
$1-\varepsilon$&$\sigma$\\
$\sigma$&$\sigma$\\
};
\draw (m.north east) rectangle (m.south west);
\draw (m.north) -- (m.south);
\draw (m.east) -- (m.west);
\coordinate (a) at ($(m.north west)!0.25!(m.north east)$);
\coordinate (b) at ($(m.north west)!0.75!(m.north east)$);
\node[above=5pt of a,anchor=base] {go};
\node[above=5pt of b,anchor=base] {not go};

\coordinate (c) at ($(m.north west)!0.25!(m.south west)$);
\coordinate (d) at ($(m.north west)!0.75!(m.south west)$);
\node[left=2pt of c,text width=1cm]  {go};
\node[left=2pt of d,text width=1cm]  {not go};

\node[above=18pt of m.north] (firm b) {Driver 2};
\node[left=1.6cm of m.west,rotate=90,align=center,anchor=center] {Driver 1};
\end{tikzpicture}
\caption{Payoff Matrix for the two-driver game.}\label{fig:twodrivers}
\end{figure}
\subsection{Cheap Talk}\label{subsec:cheaptalk}
In cheap talk, the platform discloses some information to a driver, which, given the update, updates their behavior. More concretely, there is an abstract set of \emph{messages} $m \in \mathcal M$ that the platform can send. The players receive the message and play optimally. In particular, the timeline of the game is:
\begin{enumerate}
	\item The demand $\theta$ at the area is realized.
	\item The platform decides to send messages $(m_1, m_2)$ to the drivers.
	\item The drivers decide to drive or not drive to the area.
	\end{enumerate}
We solve for perfect Bayesian (signalling) equilibria of this game.
\begin{theorem}
	In the cheap talk model, Perfect Bayesian equilibria have the following form:
	\begin{enumerate}
		\item If $\sigma < \frac12 - \varepsilon$ or $\sigma > 1-\varepsilon$, then the platform is indifferent between any message, the drivers will go to the area or not go to it.
		\item If $\sigma $
	\end{enumerate}
\end{theorem}
If the general outside option is very low ($\sigma < \frac12 - \varepsilon$) or very high ($\sigma > 1 - \varepsilon$), drivers go to the area in hope to get a ride or stay away from it, respectively, even without demand information. The platform is indifferent between any demand information given that it won't influence the driver's decisions.

For an intermediate outside option ($\frac12 - \varepsilon < \sigma < 1 - \varepsilon$), the 




In this environment, hence, if the not internalized cost from 



\subsection{Information Design}\label{subsec:infodesign}
In information design, the drivers can commit to a driver 
\subsection{Public and Private Messages}
\section{Inefficiencies}
In this environment, two sources of in
\subsection{Inefficiencies through lack of commitment}
\subsection{Inefficiencies through public information}

\section{Remedies}
Given the inefficiencies observed in the last section, some remedies for the market should be considered. We propose that 
\subsection{Commitment via third parties}
\subsection{Commitment via reputation}
\subsection{Approximate Efficiency of Public Information to Few Drivers}