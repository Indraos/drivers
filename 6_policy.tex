\chapter{Conclusion and Policy Implications}\label{chap:policy}

%TODO: Incorporate this structure
%- The commitment challenge
%- Cheap talk is a problem
%- Commitment is needed
%- Commitment needs understanding
%- External commitment is hard
%- Reputation needs to be improved
%- Open communicaiton helpful
%- Alternative: External aggregators
%
%
%- the equity challenge
%- make decisions on who gets which information
%- will become more important in the future
%- Important difference: Trip level vs. aggregate level
%
%
%- engineering challenges
%- Information design in code as a transparency problem
%- Determination of fair information design in comparison to fair matching
%- Data Access to platforms




This thesis studied the relevance and optimal design of information provided to ridesourcing drivers. \autoref{chap:jakarta} showed that besides earnings maximization, drivers also take into account cultural and infromational considerations in their labor supply decisions. \autoref{chap:chengdu} showed that there are high potential earnings gains from additional information. We then went on to study commitment and privacy of messages as two challenges. In this last chapter, we consider three stakeholder groups, platforms, regulators and transportation engineers with policy recommendations and additional complications arising out of this work. 
\section{The Commitment Challenge for Platforms}
We observed in \autoref{chap:infodesign} that the limited commitment of the platform to providing different platform can lead to not information being transmitted and we proposed reputation 
\section{The Fairness Challenge for Transportation Regulators}
\section{Challenges for Transportation Engineers}
The above two challenges give interesting challenges for transportation engineers.
\subsection{Allowing for Commitment without Harming Businesses}

\subsection{Managing Fairness in Information Provision}