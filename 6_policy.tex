\chapter{Conclusion and Policy Implications}\label{chap:policy}

%TODO: Incorporate this structure
%- The commitment challenge
%- Cheap talk is a problem
%- Commitment is needed
%- Commitment needs understanding
%- External commitment is hard
%- Reputation needs to be improved
%- Open communicaiton helpful
%- Alternative: External aggregators
%
%
%- the equity challenge
%- make decisions on who gets which information
%- will become more important in the future
%- Important difference: Trip level vs. aggregate level
%
%
%- engineering challenges
%- Information design in code as a transparency problem
%- Determination of fair information design in comparison to fair matching
%- Data Access to platforms




This thesis studied the relevance and optimal design of information provided to ridesourcing drivers. \autoref{chap:jakarta} showed that besides earnings maximization, drivers also take into account cultural and infromational considerations in their labor supply decisions. \autoref{chap:chengdu} showed that there are high potential earnings gains from additional information. We then went on to study commitment and privacy of messages as two challenges. In this last chapter, we consider three stakeholder groups, platforms, regulators and transportation engineers with policy recommendations and additional complications arising out of this work. 
\section{The Commitment Challenge for Platforms}
We observed in \autoref{chap:infodesign} that the limited commitment of the platform to providing different platform can lead to not information being transmitted and we proposed reputation 
\section{Challenges for Transportation Regulators}
Before thinking about regulation of drivers 


\section{Challenges for Transportation Engineers}
The above two challenges give interesting challenges for transportation engineers.

%The follwoign is from conversations with Steve Parrott
% - bundling additional payments.
% - sophisticated from platform (not taking into account driver)
%
%
%Market size
%
%Austin: Uber pushes that there non-going stuff (other company came in... no competition)
%
%Chicago, Boston, LA, Houston, Dallas: You need to provide us data. License needed to operate in city.
%
%Taxi market was sizeable: Tight regulation. Taxi companies need to give data: Complete informaiton on drivers. Black car fee, driving this.
%
%Right after completion of trip, NYC did this.
%
%TLC: Had experience working on this. Can regulate taxis + incomes of drivers. Charter based provision. comparable data.
%
%Chicago and other cities: No requirements.
%
%Many inefficiencies in Uber: Minimize what they pay drivers. Flood the market with drivers.
%
%Economically inefficient. Drivers are paid badly.
%
%Companies have good demand for their platform.
%
%
%
%
%Q: Do the platforms know what is needed?
%
%Good demand models on their own.
%
%
%TLC never said to match driver demand. There is not too much demand.
%
%
%TLC talking about that. COVID has been different. Before pandemic there was a lot concern about congestion. State has a lot of congestion pricing. How to more efficiently utilize street space -- freight movement.
%
%NYC DOT: Congestion charge report.
%
%Uber and Lyft are regulated at the local model. 
%
%State level is more politically compliant....California---
%
%New York city charter...
\subsection{Managing Fairness in Information Provision}