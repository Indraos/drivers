\chapter{Conclusion and Policy Implications}\label{chap:policy}

%TODO: Incorporate this structure
%
%
%- the equity challenge
%- make decisions on who gets which information
%- will become more important in the future
%- Important difference: Trip level vs. aggregate level
%
%
%- engineering challenges
%- Information design in code as a transparency problem
%- Determination of fair information design in comparison to fair matching
%- Data Access to platforms




This thesis studied the relevance and optimal design of information provided to ridesourcing drivers. \autoref{chap:jakarta} showed that besides earnings maximization, drivers also take into account cultural and infromational considerations in their labor supply decisions. \autoref{chap:chengdu} showed that there are high potential earnings gains from additional information. We then went on to study commitment and privacy of messages as two challenges. In this last chapter, we consider three stakeholder groups, platforms, regulators and transportation engineers with policy recommendations and additional complications arising out of this work. 
\section{The Commitment Challenge for Platforms}
We observed in \autoref{chap:infodesign} that the limited commitment of the platform to providing different platform can lead to drivers ignoring information.


\subsection{Employing some drivers}
In the current payment model for drivers, the TNC does not have incentives to limit the amount of drivers in a particular area. While

\subsection{Mechanical Commitment and Explainability}
 - reputation
 - audits
 - 
\subsection{External Aggregators}
 - Others can take this burden.
 
 \subsection{Bundling with Payments}





\section{Challenges for Transportation Regulators in the Case of New York}
In this section, we first outline the information access of different regulators and then put into perspective potential interventions 


When implementing this work into policy, much of

The case of Austin, Texas, sheds a light on limited possibilities of regulattion. Uber and Lyft left the market, others came in. 

New York has it in its City Charter. But other urban areas might benefit as well (Chicago, Boston, LA, Houston, Dallas)

An important aspect in the special situation of New York is its history of taxi regulation.

Federal solution to data access might be possible. 

There are good demand models, and even TLC could get a good demand model (personal communication with James Parrott). 

Matching driver demand possible, but has never happened in times of taxi. 

Congestion charge report might lead to more movement in NY.

These are other externalities, but might help.

Important question on how elastic labor supply from drivers really is.


%Right after completion of trip, NYC did this.
%
%TLC: Had experience working on this. Can regulate taxis + incomes of drivers. Charter based provision. comparable data.
%
%Chicago and other cities: No requirements.
%
%Many inefficiencies in Uber: Minimize what they pay drivers. Flood the market with drivers.
%
%Economically inefficient. Drivers are paid badly.
%%
%
%Q: Do the platforms know what is needed?
%
%Good demand models on their own.
%%
%TLC talking about that. COVID has been different. Before pandemic there was a lot concern about congestion. State has a lot of congestion pricing. How to more efficiently utilize street space -- freight movement.
%
%NYC DOT: Congestion charge report.
%
%Uber and Lyft are regulated at the local model. 
%
%State level is more politically compliant....California---
%
%New York city charter...



\section{Challenges for Transportation Engineers}
The above two challenges give interesting challenges for transportation engineers.


\subsection{Managing Fairness in Information Provision}