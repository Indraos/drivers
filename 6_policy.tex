\chapter{Conclusion and Policy Recommendations}\label{chap:policy}
\epigraph{There’s fragmentation and a lack of transparency of information on what’s happening in real time and what’s going to happen. We’re connecting bread crumbs across data areas}{Ryan Green\\as quoted in \cite{Weed2019}}
This thesis studied the relevance and optimal design of information provided to platform drivers. Our argument proceeded in three steps: \autoref{chap:jakarta} showed that drivers deviate from earnings maximization in their labor supply decisions. \autoref{chap:chengdu} showed that there are high potential earnings gains from information on demand. We found in \autoref{chap:infodesign} that a lack of commitment and public information decrease platform and driver earning when drivers are strategic. In this final chapter, we consider three stakeholder groups, platforms, regulators and transportation engineers and list open problems and recommendations for each of these groups.
\section{The Commitment Challenge for Platforms}
We observed in \autoref{chap:infodesign} that the limited commitment of the platform can lead to, in equilibrium, limits to how much information the platform can transmit such that drivers will act on the information. More specifically, the platform's commitment problem arises when it has high demand and, in an optimal information design, would show the information of high demand only to a subset of drivers. Without commitment, the platform has incentives to reveal this information to more drivers, which would maximize the likelihood that a rider is picked up, but, in turn, makes it less attractive for drivers to follow this recommendation.

As the software underlying a TNC does mostly not change, the platform has an opportunity to mechanistically commit to a particular information structure. To reach this goal, however, it is crucial that drivers \emph{understand} this commitment. There are several ways to give such asymmetry, which we outline next, in order of decreasing applicability inside of the platform.
\subsection{Designating some drivers as informationally advantaged}
If some drivers are openly designated as having access to additional information compared to other drivers, commitment to only giving some drivers information is transparent and easily communicable with drivers. Whether linked to performance on the platform or not, the main problem of a lack of commitment as identified in \autoref{chap:infodesign} disappear if the drivers being recommended to go to a high-demand area do not foresee the congestion in this place.

\subsection{Audits}
A second opportunity for platforms is to achieve their commitment is via audits and reporting from trustworthy sources. This could come, for example, via a publication of the numbers of drivers informed of high demand in an area.
\subsection{External Aggregators}
A third approach is what we observed in the marketplace already in \autoref{chap:intro}. Third parties such as Gridwise or Surge aggregate and provide information. In contrast to a TNC, these companies do, depending on their business model, not have incentives that are misaligned with platform drivers and can hence provide optimal information: The question becomes one of optimal cooperative, and not strategic information provision.

We highlight that in our analysis, we disregard the pricing aspect of the platform design problem. Pricing in forms of costly signaling, such as paying small amounts to drivers to go to a particular area, could alleviate commitment challenges as well, but are outside of the scope of this thesis.
\section{Challenges for Transportation Regulators}
As a prerequisite for understanding the regulation of information provision, understanding the informational needs of a regulator is crucial. 

The access to information from TNCs is regulated under different legislation. New York, as an example, gives their regulator, the Taxi and Limousine Commission (TLC) particularly strong powers: Section 2302 of the Charter of New York City empowers them to information access to origin-destination pairs and fares. In its executive practice, the TLC publishes data on fares set by transportation platforms and has access to disaggregated data. Importantly in the case of New York City, the power of the regulator comes from times pre-dating TNCs, when the regulator purely regulated taxis and limousines.

Other cities have different regulations, and might face platform emigration when trying to get more data access. The case of Austin, that faced emigration by Uber and Lyft when trying to introduce driver background checks, Texas showed that potential regulation might lead to platforms abandoning some areas, compare \cite{Zeitlin2019}. 

The informational demands on regulating information to drivers are much more stringent than regulation on earnings or matching: In addition to trips and fares, regulators need to get access to which information is shown to drivers, and, hence, insights into the platform.

We propose, in three steps, a path forward that would allow regulation of information provided to drivers.
\subsection{Information Provision as a Preliminary Agreement}\label{subsec:}
A first important step is that, without changing the payment structure of TNCs (no payment in phase 2), to re-classify information on demand and offers to drivers as parts of a negotiation between the platform and riders. This can have legal implications by making statements of the platform subject to pre-contractual liability, compare \cite{Society2018}. In particular, it would allow the driver to contest the veracity of demand information.

\subsection{Using Existing Information Access to Track Information Provision to Drivers}
The classification of information provision to drivers as pre-contractual negotiation would also allow existing regulators to require more information by platforms. It could then be that the list of drivers receiving information, the \emph{targeted} drivers, are shown along with the information.

\subsection{Trip Level vs. Aggregate Level Fairness}
Having access to data on information provision to platform drivers would  allow to test one of the main challenges outlined in our theoretical analysis in \autoref{chap:infodesign}: Asymmetry. Regulators would be able to regulate on which drivers get which demand information at which time.

\section{Challenges for Transportation Engineers}
The challenges for TNCs and regulators come with challenges for transportation engineers, with which we conclude this thesis. 

\subsection{Dynamic Rebalancing with Information Constraints}
Taking into account the effect of information on the reaction of drivers gives rise to challenges for operations researchers. As we saw in \autoref{chap:infodesign}, information provision to drivers needs to balance the expected congestion caused by information with the likelihood of satisfying a demand for rides. Varying between revenue maximizing and welfare maximizing platforms in information provision might give rise to interesting parameterized algorithmic challenges.

\subsection{Information Design as Market Design}
A second problem for engineers is the integration of information into pricing. Effectively targeting information to drivers with payments incentivizing them might be more efficient than information design alone. The use of combined pricing-information designs might help the functioning of urban mobility systems.

\subsection{Data Specifications}
To allow for effective regulation, the development of exchange formats for information provision is a relevant challenge. As origin-destination pairs, fares and time stamps are standard data structures  for matching and pricing problems, a data format for recommendations to drivers, might allow for the inclusion of such information into high-performance algorithms.

Depending on the development and ownership structure of Autonomous Vehicles (AVs), such specifications and information designs might even be relevant in a future without (human) drivers. Coordinating the demand of AVs through proper information design requires effective data specifications to track them.

\subsection{Operationalizing Fairness in Information Provision}
A last challenge is the operationalization of fairness in information provision and its interlinkage with pricing and matching. Describing the expected earnings for different groups based on their differential treatment by a TNC algorithm, be it in terms of information, matching, or pricing can allow policymakers and the public to discuss equity issues and distributional consequences of platforms, in addition to other (environmental and congestion, compare \cite{Diao2021}) externalities of TNCs.

We leave these challenges for the future, in hope of informed and efficient urban mobility environments. 